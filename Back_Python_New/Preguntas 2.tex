\documentclass{article}
\begin{document}

% ==================================================================
% PREGUNTA 1
% Respuesta: a), Nivel de dificultad: 1
% ==================================================================
\begin{question}{1}{logica}{1}{a}{
\\\textbf{Suponga que:}\\
\(p =\) “Llovió ayer por la noche.”\\
\(q =\) “Se prendieron los rociadores...”\\
\\[1mm]
\textbf{Basado en lo anterior...}\\
$$\neg p$$

\begin{enumerate}
    \item a) No llovió ayer por la noche.
    \item b) Se prendieron los rociadores...
    \item c) ...
\end{enumerate}
}
\end{question}

\begin{question}{2}{logica}{1}{d}{%
\\\textbf{Seleccione cuál de las siguientes oraciones \textbf{es} una proposición:}\\
\begin{enumerate}
    \item a) ¿Es 2 un número primo?
    \item b) Muestre que 2 es un número primo.
    \item c) Esta proposición es falsa.
    \item d) Pablo va a la universidad.
\end{enumerate}
}
\end{question}

\begin{question}{3}{logica}{1}{c}{%
\\\textbf{Suponga que ha contado las monedas en la mesa y descubre que hay exactamente 3.}\\
\textbf{¿Cuál de las siguientes proposiciones es \textbf{verdadera}?}\\

\begin{enumerate}
    \item a) Hay 4 monedas en la mesa.
    \item b) Hay 2 monedas en la mesa.
    \item c) Hay 3 monedas en la mesa.
    \item d) No hay monedas en la mesa.
\end{enumerate}
}
\end{question}

\begin{question}{4}{logica}{1}{a}{%
\\\textbf{Suponga que:}\\
\(p =\) "Llovió ayer por la noche."\\
\(q =\) "Se prendieron los rociadores de pasto ayer por la noche."\\
\(r =\) "El pasto estaba mojado hoy por la mañana."\\
\textbf{Basado en lo anterior, traduzca al español la proposición:}\\
$$\neg p$$

\begin{enumerate}
    \item a) No llovió ayer por la noche.
    \item b) Se prendieron los rociadores de pasto ayer por la noche.
    \item c) Llovió ayer por la noche.
    \item d) El pasto estaba mojado hoy por la mañana.
\end{enumerate}
}
\end{question}

\end{document}
