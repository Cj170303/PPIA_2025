\documentclass{article}
\begin{document}

% ==================================================================
% PREGUNTA 1
% (Tema: lógica, dif:1, res:d, week:1)
% ==================================================================
\begin{question}{1}{logica}{1}{d}{1}{
\textbf{Seleccione cuál de las siguientes oraciones \textbf{es} una proposición:}\\

\begin{enumerate}
    \item a) ¿Es 2 un número primo?
    \item b) Muestre que 2 es un número primo.
    \item c) Esta proposición es falsa.
    \item d) Pablo va a la universidad.
\end{enumerate}
}
\end{question}

% ==================================================================
% PREGUNTA 2
% (Tema: lógica, dif:1, res:c, week:1)
% ==================================================================
\begin{question}{2}{logica}{1}{c}{1}{
\textbf{Suponga que ha contado las monedas en la mesa y descubre que hay exactamente 3.}\\
\textbf{¿Cuál de las siguientes proposiciones es \textbf{verdadera}?}\\

\begin{enumerate}
    \item a) Hay 4 monedas en la mesa.
    \item b) Hay 2 monedas en la mesa.
    \item c) Hay 3 monedas en la mesa.
    \item d) No hay monedas en la mesa.
\end{enumerate}
}
\end{question}

% ==================================================================
% PREGUNTA 3
% (Tema: lógica, dif:1, res:a, week:2)
% ==================================================================
\begin{question}{3}{logica}{1}{a}{2}{
\textbf{Suponga que:}\\
\(p =\) "Llovió ayer por la noche."\\
\(q =\) "Se prendieron los rociadores de pasto ayer por la noche."\\
\(r =\) "El pasto estaba mojado hoy por la mañana."\\
\textbf{Basado en lo anterior, traduzca al español la proposición:}\\
$$\neg p$$

\begin{enumerate}
    \item a) No llovió ayer por la noche.
    \item b) Se prendieron los rociadores de pasto ayer por la noche.
    \item c) Llovió ayer por la noche.
    \item d) El pasto estaba mojado hoy por la mañana.
\end{enumerate}
}
\end{question}

% ==================================================================
% PREGUNTA 4
% (Tema: lógica, dif:1, res:b, week:2)
% ==================================================================
\begin{question}{4}{logica}{1}{b}{2}{
\textbf{Suponga que:}\\
\(p =\) "Llovió ayer por la noche."\\
\(q =\) "Se prendieron los rociadores de pasto ayer por la noche."\\
\(r =\) "El pasto estaba mojado hoy por la mañana."\\
\textbf{Basado en lo anterior, traduzca al español la proposición:}\\
$$
r \land \neg p
$$

\begin{enumerate}
    \item a) El pasto estaba mojado hoy por la mañana o no llovió ayer por la noche.
    \item b) El pasto estaba mojado hoy por la mañana y no llovió ayer por la noche.
    \item c) No se prendieron los rociadores de pasto ayer por la noche y llovió ayer por la noche.
    \item d) El pasto no estaba mojado hoy por la mañana, pero llovió ayer por la noche.
\end{enumerate}
}
\end{question}

% ==================================================================
% PREGUNTA 5
% (Tema: lógica, dif:1, res:c, week:3)
% ==================================================================
\begin{question}{5}{logica}{1}{c}{3}{
\textbf{Suponga que:}\\
\(p =\) "Llovió ayer por la noche."\\
\(q =\) "Se prendieron los rociadores de pasto ayer por la noche."\\
\(r =\) "El pasto estaba mojado hoy por la mañana."\\
\textbf{Basado en lo anterior, traduzca al español la proposición:}\\
$$
\neg r \lor p \lor q
$$

\begin{enumerate}
    \item a) El pasto no estaba mojado hoy por la mañana y llovió ayer por la noche y se prendieron los rociadores de pasto ayer por la noche.
    \item b) No llovió ayer por la noche y el pasto no estaba mojado hoy por la mañana.
    \item c) El pasto no estaba mojado hoy por la mañana, o llovió ayer por la noche, o se prendieron los rociadores de pasto ayer por la noche.
    \item d) El pasto estaba mojado hoy por la mañana, pero no se prendieron los rociadores de pasto ayer por la noche.
\end{enumerate}
}
\end{question}

% ==================================================================
% PREGUNTA 6
% (Tema: lógica, dif:2, res:b, week:3)
% ==================================================================
\begin{question}{6}{logica}{2}{b}{3}{
\textbf{Suponga que:}\\
\(p =\) "Usted tiene gripa."\\
\(q =\) "Usted no vino al último examen."\\
\(r =\) "Usted pasa el curso."\\
\textbf{Basado en lo anterior, traduzca al español la proposición:}\\
$$
(p \land \neg r)\lor (q \land \neg r)
$$

\begin{enumerate}
    \item a) Usted tiene gripa y pasa el curso, o usted no vino al último examen y pasa el curso.
    \item b) Usted tiene gripa y no pasa el curso, o usted no vino al último examen y no pasa el curso.
    \item c) Usted tiene gripa o usted pasa el curso, pero no vino al último examen.
    \item d) Usted no tiene gripa y no vino al último examen; además, no pasa el curso.
\end{enumerate}
}
\end{question}

% ==================================================================
% PREGUNTA 7
% (Tema: lógica, dif:2, res:a, week:4)
% ==================================================================
\begin{question}{7}{logica}{2}{a}{4}{
\textbf{Suponga que:}\\
\(p =\) "Usted tiene gripa."\\
\(q =\) "Usted no vino al último examen."\\
\(r =\) "Usted pasa el curso."\\
\textbf{Basado en lo anterior, traduzca al español la proposición:}\\
$$
(p \land q)\lor(\neg q \land r)
$$

\begin{enumerate}
    \item a) Usted tiene gripa y no vino al último examen, o usted sí vino al último examen y pasa el curso.
    \item b) Usted no tiene gripa y no vino al último examen, o usted sí vino al último examen y pasa el curso.
    \item c) Usted tiene gripa y vino al último examen, pero no pasa el curso.
    \item d) Usted no vino al último examen y pasa el curso, pero no tiene gripa.
\end{enumerate}
}
\end{question}

% ==================================================================
% PREGUNTA 8
% (Tema: conjuntos, dif:1, res:d, week:5)
% ==================================================================
\begin{question}{8}{conjuntos}{1}{d}{5}{
\textbf{Suponga que el universo \(U\) está conformado por estudiantes, administrativos y profesores de la Facultad de Economía.}\\
\textbf{Definimos:}
\[
R = \{x \in U \mid x \text{ recibió un regalo por CUPICONSEFE}\},\quad
E = \{x \in U \mid x \text{ envió un regalo por CUPICONSEFE}\}.
\]
\textbf{Interprete la siguiente afirmación:}
\[
R \cap E \neq \emptyset
\]

\begin{enumerate}
    \item a) No hay personas que hayan recibido y enviado regalos al mismo tiempo.
    \item b) Al menos una persona que recibió un regalo no lo envió.
    \item c) Todas las personas enviaron y recibieron regalos.
    \item d) Al menos una persona envió y recibió un regalo.
\end{enumerate}
}
\end{question}

\end{document}
